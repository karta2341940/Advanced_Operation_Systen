\documentclass[10pt, answers]{exam} 
%\documentclass[12pt, addpoints, answers]{exam} 
%\addpoints
%\noaddpoints
%\answers
\usepackage{amsmath}
\usepackage{xcolor}
\usepackage{listings}
%\lstset
%{ %Formatting for code in appendix
%    language=c,
%    basicstyle=\footnotesize,
%    numbers=left,
%    stepnumber=1,
%    showstringspaces=false,
%    tabsize=1,
%    breaklines=true,
%    breakatwhitespace=false,
%}

\usepackage{courier}
\lstset{
    language=C,
    basicstyle=\footnotesize,
    numbers=left,
    stepnumber=1,
%    showstringspaces=false,
%    tabsize=1,
%   breaklines=true,
    breakatwhitespace=false,    
%    basicstyle=\footnotesize\ttfamily, % Default font
    % numbers=left,              % Location of line numbers
    numberstyle=\tiny,          % Style of line numbers
    % stepnumber=2,              % Margin between line numbers
    numbersep=5pt,              % Margin between line numbers and text
%    tabsize=2,                  % Size of tabs
    extendedchars=true,
    breaklines=true,            % Lines will be wrapped
%    keywordstyle=\color{red},
    frame=b,
    % keywordstyle=[1]\textbf,
    % keywordstyle=[2]\textbf,
    % keywordstyle=[3]\textbf,
    % keywordstyle=[4]\textbf,   \sqrt{\sqrt{}}
%    stringstyle=\color{white}\ttfamily, % Color of strings
    stringstyle=\ttfamily, % Color of strings
    showspaces=false,
    showtabs=false,
    xleftmargin=17pt,
    framexleftmargin=17pt,
    framexrightmargin=5pt,
    framexbottommargin=4pt,
    % backgroundcolor=\color{lightgray},
    showstringspaces=true
}
%%%\lstloadlanguages{ % Check documentation for further languages ...
%%%     % [Visual]Basic,
%%%     % Pascal,
%%%     C,
%%%     C++,
%%%     % XML,
%%%     % HTML,
%%%     Java
%%%}
% \DeclareCaptionFont{blue}{\color{blue}} 

% \captionsetup[lstlisting]{singlelinecheck=false, labelfont={blue}, textfont={blue}}
\usepackage{caption}
\DeclareCaptionFont{white}{\color{white}}
\DeclareCaptionFormat{listing}{\colorbox[cmyk]{0.43, 0.35, 0.35,0.01}{\parbox{\textwidth}{\hspace{15pt}#1#2#3}}}
\captionsetup[lstlisting]{format=listing,labelfont=white,textfont=white, singlelinecheck=false, margin=0pt, font={bf,footnotesize}}




\usepackage{hyperref}
\usepackage{graphicx}
\usepackage{subfigure}
\usepackage{multirow}


\newcommand{\coursename}{Advanced Operating Systems}
\newcommand{\semester}{Spring 2023}
\newcommand{\hwtitle}{HW\#3}
\newcommand{\studentname}{MengXian,Du}
\newcommand{\studentID}{CBB108047}

\newcommand{\answer}{\\~\textbf{Answer:}\space}

\pagestyle{head} 
\pagestyle{headandfoot}
\extraheadheight{1in}
\firstpageheader{}
{\begin{large}\hwtitle \\
\coursename, \semester\end{large}\\~\\
\studentname\space(\studentID)\\
Department of Computer Science and Information Engineering\\
National Pingtung University}
{}
%\runningheadrule 
%\runningheader{}{}{}
\setlength\answerlinelength{2in}
\unframedsolutions

\begin{document}
%\noindent This homework is from Chapter 5 of OSTEP.

%1
\begin{questions} 
\setcounter{question}{0} 

\question 
First, write a simple program called null.c that creates a pointer
to an integer, sets it to NULL, and then tries to dereference it.
Compile this into an executable called null. What happens when you
run this program?

\begin{solution}
Please refer to List 1 (null.c):

\lstinputlisting[label=code-null.c, caption=null.c]{../null.c}

Its execution results are as follows:

\begin{lstlisting}[language=bash]
Start
Segmentation fault (core dumped)
\end{lstlisting}

As we can see the program is terminated with a segmentation fault.

So we know that if a variable doesn't allocated memory space,
it will cause a segmentation fault when it is dereferenced.

But if we need to reference it.
It's ok to reference it.
\end{solution}



\end{questions}

%2
\begin{questions} 
\setcounter{question}{1} 
\question 
Next, compile this program with symbol information included (with
the -g flag). Doing so let’s put more information into the executable,
enabling the debugger to access more useful information about variable 
names and the like. Run the program under the debugger by typing gdb null 
and then, once gdb is running, typing run. What does gdb show you?

\begin{solution}
    
Its execution results are as follows:
\begin{lstlisting}[language=bash]
(gdb) run
Starting program: /home/it/Desktop/Advanced_Operation_Systen/hw3/null 
[Thread debugging using libthread_db enabled]
Using host libthread_db library "/lib/x86_64-linux-gnu/libthread_db.so.1".
Start
The address of p is (nil)

Program received signal SIGSEGV, Segmentation fault.
0x00005555555551ab in main () at null.c:16
16          printf("The value of p is %d\n", *p);
\end{lstlisting}
The gdb shows that the program is terminated with a segmentation fault.

\end{solution}
\end{questions}


%3
\begin{questions} 
\setcounter{question}{2} 
\question 
Finally, use the valgrind tool on this program. We’ll use the memcheck
tool that is a part of valgrind to analyze what happens. Run
this by typing in the following: valgrind --leak-check=yes
null. What happens when you run this? Can you interpret the
output from the tool?
\begin{solution}
    
Its execution results are as follows:
\begin{lstlisting}[language=bash]
==20372== Memcheck, a memory error detector
==20372== Copyright (C) 2002-2017, and GNU GPL'd, by Julian Seward et al.
==20372== Using Valgrind-3.18.1 and LibVEX; rerun with -h for copyright info
==20372== Command: ./null
==20372== 
Start
The address of p is (nil)
==20372== Invalid read of size 4
==20372==    at 0x1091AB: main (null.c:16)
==20372==  Address 0x0 is not stack'd, malloc'd or (recently) free'd
==20372== 
==20372== 
==20372== Process terminating with default action of signal 11 (SIGSEGV)
==20372==  Access not within mapped region at address 0x0
==20372==    at 0x1091AB: main (null.c:16)
==20372==  If you believe this happened as a result of a stack
==20372==  overflow in your program's main thread (unlikely but
==20372==  possible), you can try to increase the size of the
==20372==  main thread stack using the --main-stacksize= flag.
==20372==  The main thread stack size used in this run was 8388608.
==20372== 
==20372== HEAP SUMMARY:
==20372==     in use at exit: 1,024 bytes in 1 blocks
==20372==   total heap usage: 1 allocs, 0 frees, 1,024 bytes allocated
==20372== 
==20372== LEAK SUMMARY:
==20372==    definitely lost: 0 bytes in 0 blocks
==20372==    indirectly lost: 0 bytes in 0 blocks
==20372==      possibly lost: 0 bytes in 0 blocks
==20372==    still reachable: 1,024 bytes in 1 blocks
==20372==         suppressed: 0 bytes in 0 blocks
==20372== Reachable blocks (those to which a pointer was found) are not shown.
==20372== To see them, rerun with: --leak-check=full --show-leak-kinds=all
==20372== 
==20372== For lists of detected and suppressed errors, rerun with: -s
==20372== ERROR SUMMARY: 1 errors from 1 contexts (suppressed: 0 from 0)
Segmentation fault (core dumped)
\end{lstlisting}
When I run 'null' with valgrind, it shows that it can't 
read the memory that it's size is 4.

Address 0x0 is not stack'd, malloc'd or (recently) free'd means that
the address 0x0 is not allocated and recently freed. Also the program 
is terminated with a SIGSEGV signal which means a process executes a 
invalid memory reference.

And we can see at line 23 shows that the program has a memory block in use at exit.

More see :\url{https://stackoverflow.com/questions/3840582/still-reachable-leak-detected-by-valgrind}

\end{solution}
\end{questions}


%4
\begin{questions} 
    \setcounter{question}{3} 
    
    \question 
    Write a simple program that allocates memory using malloc() but
    forgets to free it before exiting. What happens when this program
    runs? Can you use gdb to find any problems with it? How about
    valgrind (again with the --leak-check=yes flag)?

    
    \begin{solution}
    Please refer to List 1 (q4.c):
    
    \lstinputlisting[label=code-q4.c, caption=q4.c]{../q4.c}
    
    Its execution results are as follows:
    \begin{lstlisting}[language=bash]
Start
The address of p is 0x55f62934c2a0
The value of p is 0
End
    \end{lstlisting}

    The result use gdb is as follows:
    \begin{lstlisting}[language=bash]
(gdb) run
Starting program: /home/it/Desktop/Advanced_Operation_Systen/hw3/q4
[Thread debugging using libthread_db enabled]
Using host libthread_db library "/lib/x86_64-linux-gnu/libthread_db.so.1".
Start
The address of p is 0x5555555592a0
The value of p is 0
End
[Inferior 1 (process 22758) exited normally]
    \end{lstlisting}

    The result use valgrind is as follows:
    \begin{lstlisting}[language=bash]
==22981== Memcheck, a memory error detector
==22981== Using Valgrind-3.18.1 and LibVEX; rerun with -h for copyright info
==22981== Copyright (C) 2002-2017, and GNU GPL'd, by Julian Seward et al.
==22981== Command: ./q4
==22981==
Start
The address of p is 0x4a96040
==22981== Conditional jump or move depends on uninitialised value(s)
==22981==    at 0x48E1B56: __vfprintf_internal (vfprintf-internal.c:1516)
==22981==    by 0x48CB81E: printf (printf.c:33)
==22981==    by 0x1091E8: main (in /home/it/Desktop/Advanced_Operation_Systen/hw3/q4)
==22981==
==22981== Use of uninitialised value of size 8
==22981==    at 0x48C533B: _itoa_word (_itoa.c:177)
==22981==    by 0x48E0B3D: __vfprintf_internal (vfprintf-internal.c:1516)
==22981==    by 0x48CB81E: printf (printf.c:33)
==22981==    by 0x1091E8: main (in /home/it/Desktop/Advanced_Operation_Systen/hw3/q4)
==22981==
==22981== Conditional jump or move depends on uninitialised value(s)
==22981==    at 0x48C534C: _itoa_word (_itoa.c:177)
==22981==    by 0x48E0B3D: __vfprintf_internal (vfprintf-internal.c:1516)
==22981==    by 0x48CB81E: printf (printf.c:33)
==22981==    by 0x1091E8: main (in /home/it/Desktop/Advanced_Operation_Systen/hw3/q4)
==22981==
==22981== Conditional jump or move depends on uninitialised value(s)
==22981==    at 0x48E1643: __vfprintf_internal (vfprintf-internal.c:1516)
==22981==    by 0x48CB81E: printf (printf.c:33)
==22981==    by 0x1091E8: main (in /home/it/Desktop/Advanced_Operation_Systen/hw3/q4)
==22981==
==22981== Conditional jump or move depends on uninitialised value(s)
==22981==    at 0x48E0C85: __vfprintf_internal (vfprintf-internal.c:1516)
==22981==    by 0x48CB81E: printf (printf.c:33)
==22981==    by 0x1091E8: main (in /home/it/Desktop/Advanced_Operation_Systen/hw3/q4)
==22981==
The value of p is 0
End
==22981==
==22981== HEAP SUMMARY:
==22981==     in use at exit: 4 bytes in 1 blocks
==22981==   total heap usage: 2 allocs, 1 frees, 1,028 bytes allocated
==22981==
==22981== 4 bytes in 1 blocks are definitely lost in loss record 1 of 1
==22981==    at 0x4848899: malloc (in /usr/libexec/valgrind/vgpreload_memcheck-amd64-linux.so)
==22981==    by 0x10919E: main (in /home/it/Desktop/Advanced_Operation_Systen/hw3/q4)
==22981==
==22981== LEAK SUMMARY:
==22981==    definitely lost: 4 bytes in 1 blocks
==22981==    indirectly lost: 0 bytes in 0 blocks
==22981==      possibly lost: 0 bytes in 0 blocks
==22981==    still reachable: 0 bytes in 0 blocks
==22981==         suppressed: 0 bytes in 0 blocks
==22981==
==22981== Use --track-origins=yes to see where uninitialised values come from
==22981== For lists of detected and suppressed errors, rerun with: -s
==22981== ERROR SUMMARY: 6 errors from 6 contexts (suppressed: 0 from 0)

    \end{lstlisting}

    1. What happens when this program runs?
    
    Ans : The program executes normally seems not big deal.

    
    2. Can you use gdb to find any problems with it?
    
    Ans : No, gdb shows the program is terminated normally.
    
    
    3. How about valgrind (again with the --leak-check=yes flag)?
    
    Ans : Yes, valgrind shows that the program has a memory leak.
    \end{solution}
    
    
    
    \end{questions}
%5
\begin{questions} 
        \setcounter{question}{4} 
        
        \question 
        Write a program that creates an array of integers called data
        of size 100 using malloc; then, set data[100] to zero. What happens
        when you run this program? What happens when you run this
        program using valgrind? Is the program correct?

        
        \begin{solution}
        Please refer to List 1 (q5.c):
        
        \lstinputlisting[label=code-q5.c, caption=q5.c]{../q5.c}
        
        Its execution results are as follows:
        
        \begin{lstlisting}[language=bash]
data[100] = 0
        \end{lstlisting}
        
        Valgrind's execution results are as follows:
        
        \begin{lstlisting}[language=bash]
==24012== Memcheck, a memory error detector
==24012== Copyright (C) 2002-2017, and GNU GPL'd, by Julian Seward et al.
==24012== Using Valgrind-3.18.1 and LibVEX; rerun with -h for copyright info
==24012== Command: ./q5
==24012==
==24012== Invalid write of size 4
==24012==    at 0x10918D: main (in /home/it/Desktop/Advanced_Operation_Systen/hw3/q5)
==24012==  Address 0x4a961d0 is 0 bytes after a block of size 400 alloc'd
==24012==    at 0x4848899: malloc (in /usr/libexec/valgrind/vgpreload_memcheck-amd64-linux.so)
==24012==    by 0x10917E: main (in /home/it/Desktop/Advanced_Operation_Systen/hw3/q5)
==24012==
==24012== Invalid read of size 4
==24012==    at 0x10919D: main (in /home/it/Desktop/Advanced_Operation_Systen/hw3/q5)
==24012==  Address 0x4a961d0 is 0 bytes after a block of size 400 alloc'd
==24012==    at 0x4848899: malloc (in /usr/libexec/valgrind/vgpreload_memcheck-amd64-linux.so)
==24012==    by 0x10917E: main (in /home/it/Desktop/Advanced_Operation_Systen/hw3/q5)
==24012==
data[100] = 0
==24012==
==24012== HEAP SUMMARY:
==24012==     in use at exit: 400 bytes in 1 blocks
==24012==   total heap usage: 2 allocs, 1 frees, 1,424 bytes allocated
==24012==
==24012== 400 bytes in 1 blocks are definitely lost in loss record 1 of 1
==24012==    at 0x4848899: malloc (in /usr/libexec/valgrind/vgpreload_memcheck-amd64-linux.so)
==24012==    by 0x10917E: main (in /home/it/Desktop/Advanced_Operation_Systen/hw3/q5)
==24012==
==24012== LEAK SUMMARY:
==24012==    definitely lost: 400 bytes in 1 blocks
==24012==    indirectly lost: 0 bytes in 0 blocks
==24012==      possibly lost: 0 bytes in 0 blocks
==24012==    still reachable: 0 bytes in 0 blocks
==24012==         suppressed: 0 bytes in 0 blocks
==24012==
==24012== For lists of detected and suppressed errors, rerun with: -s
==24012== ERROR SUMMARY: 3 errors from 3 contexts (suppressed: 0 from 0)
        \end{lstlisting}
        
        1. What happens when you run this program?
        
        Ans : The program executes normally seems good no big problems.

        2. What happens when you run this program using valgrind?

        Ans : Valgrind shows that the program has a memory leak and
        invalid read and write.

        3. Is the program correct?

        Ans : No, the program is not correct. Becaues 
        it use the memory that is not allocated.
        \end{solution}
        
        
        
        \end{questions}

%6
\begin{questions} 
    \setcounter{question}{5} 
    
    \question 
    Create a program that allocates an array of integers (as above), frees
    them, and then tries to print the value of one of the elements of
    the array. Does the program run? What happens when you use
    valgrind on it?

    
    \begin{solution}
    Please refer to List 1 (q6.c):
    
    \lstinputlisting[label=code-q6.c, caption=q6.c]{../q6.c}
    
    Its execution results are as follows:
    
    \begin{lstlisting}[language=bash]
data[0] = 1570806496
    \end{lstlisting}
    
    Valgrind's execution results are as follows:
    
    \begin{lstlisting}[language=bash]
==24375== Command: ./q6
==24375==
==24375== Invalid read of size 4
==24375==    at 0x1091B3: main (in /home/it/Desktop/Advanced_Operation_Systen/hw3/q6)
==24375==  Address 0x4a96040 is 0 bytes inside a block of size 400 free'd
==24375==    at 0x484B27F: free (in /usr/libexec/valgrind/vgpreload_memcheck-amd64-linux.so)
==24375==    by 0x1091AE: main (in /home/it/Desktop/Advanced_Operation_Systen/hw3/q6)
==24375==  Block was alloc'd at
==24375==    at 0x4848899: malloc (in /usr/libexec/valgrind/vgpreload_memcheck-amd64-linux.so)
==24375==    by 0x10919E: main (in /home/it/Desktop/Advanced_Operation_Systen/hw3/q6)
==24375==
data[0] = 0
==24375==
==24375== HEAP SUMMARY:
==24375==     in use at exit: 0 bytes in 0 blocks
==24375==   total heap usage: 2 allocs, 2 frees, 1,424 bytes allocated
==24375==
==24375== All heap blocks were freed -- no leaks are possible
==24375==
==24375== For lists of detected and suppressed errors, rerun with: -s
==24375== ERROR SUMMARY: 1 errors from 1 contexts (suppressed: 0 from 0)
    \end{lstlisting}
    
    1. Does the program run?
    
    Ans : Yes, The program can be run but the value is 
    not controllable and predictable.

    2. What happens when you use valgrind on it?

    Ans : Valgrind shows that the program has no memory leak 
    but there have a memory block has invalid read and it's size is 4 bytes.

    \end{solution}
    
    
    
    \end{questions}

%7
\begin{questions} 
    \setcounter{question}{6} 
    
    \question 
    Now pass a funny value to free (e.g., a pointer in the middle of the
    array you allocated above). What happens? Do you need tools to
    find this type of problem?

    
    \begin{solution}
    Please refer to List 1 (q7.c):
    
    \lstinputlisting[label=code-q7.c, caption=q7.c]{../q7.c}
    
    Its execution results are as follows:
    
    \begin{lstlisting}[language=bash]
free(): invalid pointer
Aborted (core dumped)
    \end{lstlisting}
    
    Valgrind's execution results are as follows:
    
    \begin{lstlisting}[language=bash]
==25997== Command: ./q7
==25997==
==25997== Invalid free() / delete / delete[] / realloc()
==25997==    at 0x484B27F: free (in /usr/libexec/valgrind/vgpreload_memcheck-amd64-linux.so)
==25997==    by 0x1091BC: main (in /home/it/Desktop/Advanced_Operation_Systen/hw3/q7)
==25997==  Address 0x4a96044 is 4 bytes inside a block of size 12 alloc'd
==25997==    at 0x4848899: malloc (in /usr/libexec/valgrind/vgpreload_memcheck-amd64-linux.so)
==25997==    by 0x10919E: main (in /home/it/Desktop/Advanced_Operation_Systen/hw3/q7)
==25997==
data[0] = 1
==25997==
==25997== HEAP SUMMARY:
==25997==     in use at exit: 12 bytes in 1 blocks
==25997==   total heap usage: 2 allocs, 2 frees, 1,036 bytes allocated
==25997==
==25997== 12 bytes in 1 blocks are definitely lost in loss record 1 of 1
==25997==    at 0x4848899: malloc (in /usr/libexec/valgrind/vgpreload_memcheck-amd64-linux.so)
==25997==    by 0x10919E: main (in /home/it/Desktop/Advanced_Operation_Systen/hw3/q7)
==25997==
==25997== LEAK SUMMARY:
==25997==    definitely lost: 12 bytes in 1 blocks
==25997==    indirectly lost: 0 bytes in 0 blocks
==25997==      possibly lost: 0 bytes in 0 blocks
==25997==    still reachable: 0 bytes in 0 blocks
==25997==         suppressed: 0 bytes in 0 blocks
==25997==
==25997== For lists of detected and suppressed errors, rerun with: -s
==25997== ERROR SUMMARY: 2 errors from 2 contexts (suppressed: 0 from 0)
    \end{lstlisting}
    
    1. What happens?
    
    Ans : The program is terminated with a SIGABRT signal which means 
    a process aborts.

    2. Do you need tools to find this type of problem?

    Ans : Yes ,I use valgrind to find this type of problem.
    We can see the output of valgrind that we do a invalid free() operation.
    It seems we can't free a memory block that in the 
    middle of a series of memory that allocated.

    \end{solution}
    
    
    
    \end{questions}

%8
\begin{questions} 
    \setcounter{question}{7} 
    
    \question 
    Try out some of the other interfaces to memory allocation. For example, create a simple vector-like data structure and related routines that use realloc() to manage the vector. Use an array to
    store the vectors elements; when a user adds an entry to the vector, use realloc() to allocate more space for it. How well does
    such a vector perform? How does it compare to a linked list? Use
    valgrind to help you find bugs.


    
    \begin{solution}
    Please refer to List 1 (q8.c):
    
    \lstinputlisting[label=code-q8.c, caption=q8.c]{../q8.c}
    
    Its execution results are as follows:
    
    \begin{lstlisting}[language=bash]
Linked List: 1
Linked List: 1 2
Linked List: 1 2 3
Linked List: 1 2
Linked List: 1
size: 1, capacity: 1 => The Value in vector are 1
size: 2, capacity: 2 => The Value in vector are 1 2
size: 3, capacity: 4 => The Value in vector are 1 2 3
size: 3, capacity: 4 => The Value in vector are 1790712579 21891 -529391540
    \end{lstlisting}
    
    Valgrind's execution results are as follows:
    
    \begin{lstlisting}[language=bash]
==31906== Conditional jump or move depends on uninitialised value(s)
==31906==    at 0x10942C: llshow (in /home/it/Desktop/Advanced_Operation_Systen/hw3/q8)
==31906==    by 0x109537: main (in /home/it/Desktop/Advanced_Operation_Systen/hw3/q8)
==31906==
Linked List: 1
==31906== Conditional jump or move depends on uninitialised value(s)
==31906==    at 0x10946F: llpush (in /home/it/Desktop/Advanced_Operation_Systen/hw3/q8)
==31906==    by 0x109548: main (in /home/it/Desktop/Advanced_Operation_Systen/hw3/q8)
==31906==
Linked List: 1 2
Linked List: 1 2 3
Linked List: 1 2
Linked List: 1
size: 1, capacity: 1 => The Value in vector are 1
size: 2, capacity: 2 => The Value in vector are 1 2
size: 3, capacity: 4 => The Value in vector are 1 2 3
==31906== Invalid read of size 4
==31906==    at 0x109381: vshow (in /home/it/Desktop/Advanced_Operation_Systen/hw3/q8)
==31906==    by 0x10961C: main (in /home/it/Desktop/Advanced_Operation_Systen/hw3/q8)
==31906==  Address 0x4a96610 is 0 bytes inside a block of size 16 free'd
==31906==    at 0x484B27F: free (in /usr/libexec/valgrind/vgpreload_memcheck-amd64-linux.so)
==31906==    by 0x10932D: vclear (in /home/it/Desktop/Advanced_Operation_Systen/hw3/q8)
==31906==    by 0x109610: main (in /home/it/Desktop/Advanced_Operation_Systen/hw3/q8)
==31906==  Block was alloc'd at
==31906==    at 0x484DCD3: realloc (in /usr/libexec/valgrind/vgpreload_memcheck-amd64-linux.so)
==31906==    by 0x1092D7: vpush (in /home/it/Desktop/Advanced_Operation_Systen/hw3/q8)
==31906==    by 0x1095F8: main (in /home/it/Desktop/Advanced_Operation_Systen/hw3/q8)
==31906==
size: 3, capacity: 4 => The Value in vector are 1 2 3
==31906==
==31906== HEAP SUMMARY:
==31906==     in use at exit: 16 bytes in 1 blocks
==31906==   total heap usage: 7 allocs, 6 frees, 1,100 bytes allocated
==31906==
==31906== 16 bytes in 1 blocks are definitely lost in loss record 1 of 1
==31906==    at 0x4848899: malloc (in /usr/libexec/valgrind/vgpreload_memcheck-amd64-linux.so)
==31906==    by 0x10951D: main (in /home/it/Desktop/Advanced_Operation_Systen/hw3/q8)
==31906==
==31906== LEAK SUMMARY:
==31906==    definitely lost: 16 bytes in 1 blocks
==31906==    indirectly lost: 0 bytes in 0 blocks
==31906==      possibly lost: 0 bytes in 0 blocks
==31906==    still reachable: 0 bytes in 0 blocks
==31906==         suppressed: 0 bytes in 0 blocks
==31906==
==31906== Use --track-origins=yes to see where uninitialised values come from
==31906== For lists of detected and suppressed errors, rerun with: -s
==31906== ERROR SUMMARY: 6 errors from 4 contexts (suppressed: 0 from 0)
    \end{lstlisting}
    
    1. How well does such a vector perform?
    
    Ans : It's perform I is very well, It easy to use and clear to know.
    This is a good way to store data.

    2. How does it compare to a linked list?

    Ans : It compare to linked list is faster then linked list.
    If we need to add a new element to the end of vector and linked list,
    the vector is faster than linked list.
    Becaues of linked list need to find the last element of linked list,
    but the vector just need to add the new element to the end of vector.
    and it has stored the size and capacity of vector.

    3. Use valgrind to help you find bugs

    \end{solution}
    
    
    
    \end{questions}

\end{document}